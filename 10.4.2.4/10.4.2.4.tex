\documentclass[12pt]{article}
\usepackage{amsmath}
\newcommand{\myvec}[1]{\ensuremath{\begin{pmatrix}#1\end{pmatrix}}}
\newcommand{\mydet}[1]{\ensuremath{\begin{vmatrix}#1\end{vmatrix}}}
\newcommand{\solution}{\noindent \textbf{Solution: }}
\providecommand{\brak}[1]{\ensuremath{\left(#1\right)}}
\providecommand{\norm}[1]{\left\lVert#1\right\rVert}
\let\vec\mathbf

\title{Quadratic Equations}
\author{ K.Hritik (kottahritik@sriprakashschools.com)}
\title{Quadratic equations}
\author{K.Hritik (kottahritik@sriprakashschools.com)}
\begin{document}
\section*{10$^{th}$ Maths - Chapter 4}
This is Problem-4 from Exercise 4.2
\begin{enumerate}
\item Find the two positive consecutive integers whose sum of squares is 365.
\end{enumerate}
\solution \\
Given Data:sum of the squares of two numbers = 365\\
Let, the positive integers be x and x+1.\\
So,According to the question.
\begin{align}
(x)^{2} + (x+1)^{2} &= 365 \\
x^{2} + x^{2} + 1 + 2x &= 365 \\
2x^{2} +2x &= 365 -1\\
2(x^{2} +x) &= 364\\
x^{2} + x - 182 &= 0
\end{align}
By formula method of finding x we get,\\
\begin{align}
x &= \frac{-b\pm\sqrt{b^2-4ac}}{2a}\\
x &= \frac{-1\pm\sqrt{1^2-4(1)(182)}}{2(1)}\\
x &= \frac{-1 \pm\sqrt{1-728}}{2}\\
x &= \frac{-1 \pm\sqrt{729}}{2}\\
x &= \frac{-1 \pm 27}{2}
\end{align}
1st condition
\begin{align}
x &=\frac{-1 + 27}{2}\\
x &=\frac{26}{2}\\ 
x &=13
\end{align}
2nd condition 
\begin{align}
x &= \frac{-1 - 27}{2}\\
x &=\frac{-28}{2}\\
x &= -14
\end{align}
since,according to the question we have to find positive integers,\\
x=13\\
x+1=13+1=14\\

\end{document}